%%%%%%%%%%%%%%%%%%%%%%%%%%%%%%%%%%%%%%%%%
% Beamer Presentation
% LaTeX Template
% Version 1.0 (10/11/12)
%
% This template has been downloaded from:
% http://www.LaTeXTemplates.com
%
% License:
% CC BY-NC-SA 3.0 (http://creativecommons.org/licenses/by-nc-sa/3.0/)
%
%%%%%%%%%%%%%%%%%%%%%%%%%%%%%%%%%%%%%%%%%

%----------------------------------------------------------------------------------------
%   PACKAGES AND THEMES
%----------------------------------------------------------------------------------------

\documentclass{beamer}

\mode<presentation> {

% The Beamer class comes with a number of default slide themes
% which change the colors and layouts of slides. Below this is a list
% of all the themes, uncomment each in turn to see what they look like.

% \usetheme{default}
% \usetheme{AnnArbor}
% \usetheme{Antibes}
% \usetheme{Bergen}
% \usetheme{Berkeley}
% \usetheme{Berlin}
% \usetheme{Boadilla}
% \usetheme{CambridgeUS}
% \usetheme{Copenhagen}
% \usetheme{Darmstadt}
% \usetheme{Dresden}
% \usetheme{Frankfurt}
% \usetheme{Goettingen}
% \usetheme{Hannover}
% \usetheme{Ilmenau}
% \usetheme{JuanLesPins}
% \usetheme{Luebeck}

\usetheme{Madrid}

% \usetheme{Malmoe}
% \usetheme{Marburg}
% \usetheme{Montpellier}
% \usetheme{PaloAlto}
% \usetheme{Pittsburgh}
% \usetheme{Rochester}
% \usetheme{Singapore}
% \usetheme{Szeged}
% \usetheme{Warsaw}

% As well as themes, the Beamer class has a number of color themes
% for any slide theme. Uncomment each of these in turn to see how it
% changes the colors of your current slide theme.

% \usecolortheme{albatross}
% \usecolortheme{beaver}
% \usecolortheme{beetle}
% \usecolortheme{crane}

\usecolortheme{dolphin}

% \usecolortheme{dove}
% \usecolortheme{fly}
% \usecolortheme{lily}
% \usecolortheme{orchid}
% \usecolortheme{rose}
% \usecolortheme{seagull}
% \usecolortheme{seahorse}
% \usecolortheme{whale}
% \usecolortheme{wolverine}

%\setbeamertemplate{footline} % To remove the footer line in all slides uncomment this line
%\setbeamertemplate{footline}[page number] % To replace the footer line in all slides with a simple slide count uncomment this line

%\setbeamertemplate{navigation symbols}{} % To remove the navigation symbols from the bottom of all slides uncomment this line
}

\usepackage{graphicx} % Allows including images
\usepackage{booktabs} % Allows the use of \toprule, \midrule and \bottomrule in tables
\usepackage[draft]{todonotes}

\usepackage[utf8]{inputenc}
\usepackage{algorithm}
\usepackage[noend]{algpseudocode}
%----------------------------------------------------------------------------------------
%   TITLE PAGE
%----------------------------------------------------------------------------------------

\title[TP3]{Métodos Numéricos \\ TP3} % The short title appears at the bottom of every slide, the full title is only on the title page

\author{Seijo, De Bortoli, Penas, Grings} % Your name
\institute[DC] % Your institution as it will appear on the bottom of every slide, may be shorthand to save space
{
% FCEN - UBA \\ % Your institution for the title page
% \medskip
% \textit{jon.seijo@gmail.com} % Your email address
}

% \date{\today} % Date, can be changed to a custom date
\date{Noviembre 2017} % Date, can be changed to a custom date

\begin{document}

\begin{frame}
\titlepage % Print the title page as the first slide
\end{frame}

% \begin{frame}
% \frametitle{Overview} % Table of contents slide, comment this block out to remove it
% \tableofcontents % Throughout your presentation, if you choose to use \section{} and \subsection{} commands, these will automatically be printed on this slide as an overview of your presentation
% \end{frame}

%----------------------------------------------------------------------------------------
%   PRESENTATION SLIDES
%----------------------------------------------------------------------------------------

%------------------------------------------------
\section{Introducción} % Sections can be created in order to organize your presentation into discrete blocks, all sections and subsections are automatically printed in the table of contents as an overview of the talk
%------------------------------------------------


\subsection{Introducción} % A subsection can be created just before a set of slides with a common theme to further break down your presentation into chunks

\begin{frame}
% \frametitle{TO DO LIST}

% \todo[inline]{Introduccion ?}
\todo[inline]{Como tomamos los datos, que usamos para implementar (python etc). me dijo fran que lo mencionaramos}
\todo[inline]{Division de datos en semanas}

\todo[inline]{Primer eje, explicacion}
\todo[inline]{Mostrar grafico de cancelaciones sin prediccion}
\todo[inline]{Intuicion de que funcion elegir. Periodos de cosenos son las semanas, etc}
\todo[inline]{Resultados}
\todo[inline]{Que pasa si filtramos por aropuerto (Miami + el otro)}
\todo[inline]{Conclusion}

\todo[inline]{Segundo eje, explicacion}
\todo[inline]{Por que filtramos por aeropuerto}
\todo[inline]{Funciones de pred, resultados}
\todo[inline]{Algo mas quiza?}
\todo[inline]{Conclusion}

\todo[inline]{Preguntas}


\end{frame}

% -----------------------------------

\begin{frame}
\frametitle{Ejemplo del comando \textbackslash item}

\begin{itemize}
    \item<1-> {Para llegar a la solución tenemos que probar diferentes posibilidades.}
    \item<2-> {Una forma de encarar el problema es pensar en las decisiones que pude haber tomado para llegar a mi posición actual.}
    \item<3-> {En nuestro caso, si estoy parado en una cierta casilla \textbf{solamente} pude haber llegado desde la casilla de \textbf{arriba} o desde la \textbf{izquierda}.}
    \item<4> {\textit{Cuidado con los bordes}. Por ejemplo, si estamos en el borde izquierdo solo pudimos venir desde arriba. Ignoremos los casos borde por unos momentos}

\end{itemize}

\end{frame}

% ------------------------------

\begin{frame}
\frametitle{Ejemplo del comando pause}

% Con la observación en mente, ya sabemos como se relacionan las casillas en un camino mínimo. \\
% \pause
Queremos encontrar el peso del camino mínimo que va desde (1,1) hasta (m,n). Pongámosle nombre a lo que buscamos.\\
\pause
\begin{block}{Definición}
f(i, j) = Peso del camino mínimo que va desde (1,1) hasta (i,j).
\end{block}
\pause
$ $\newline
La solución a nuestro problema es f(m, n).
\end{frame}





%------------------------------------------------

\begin{frame}
\Huge{\centerline{Preguntas}}
\end{frame}

%----------------------------------------------------------------------------------------

\end{document}